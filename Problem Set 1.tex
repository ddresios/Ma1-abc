\documentclass[11pt]{article}

\usepackage{amsfonts,amsmath,amssymb,latexsym,color,epsfig}
\setlength{\textheight}{22.5cm} \setlength{\textwidth}{6.7in}
\setlength{\topmargin}{0pt} \setlength{\evensidemargin}{1pt}
\setlength{\oddsidemargin}{1pt} \setlength{\headsep}{10pt}
\setlength{\parskip}{1mm} \setlength{\parindent}{0mm}
\renewcommand{\baselinestretch}{1.03}

\newtheorem{theorem}{Theorem}
\newtheorem{conjecture}{Conjecture}
\newenvironment{proof}[1]
      {\medskip\noindent{\bf Proof #1}\hspace{1mm}}
      {\hfill$\Box$\medskip}


\title{\vspace{-0.7cm}Ma 1A - Problem Set 1}
\date{}

\begin{document}

\maketitle

\begin{enumerate}

\item
Prove that $x + \frac 1x \geq 2$ for all $x > 0$ and use this result to show that $\frac{a+b}{2} \geq \sqrt{ab}$ for all $a, b > 0$.

\bigskip 
\textbf{Proof: } First show that $x + \frac 1x \geq 2$ for all $x > 0$. Start by multiplying both sides of the inequality by $x$, which is ok because we only care about $x\geq 0$ and don't have to worry about sign switching or multiplying by a negative number: 
\[x^2 + 1 \geq 2x\]
Subtract $2x$ from both sides: 
\[x^2 -2x + 1 \geq 0 \]
Now let's find the minimum value of the left side of the inequality. Take the derivative and set it equal to zero: 
\[\tfrac{d}{dx} (x^2-2x+1 )= 2x -2 = 0 \]
\[2x = 2 \]
\[x = 1 \]
Thus we know that the minimum value of $x^2 - 2x + 1$ occurs at $x = 1$. Plug in $x = 1$ to the inequality to find the minimum value: 
\[(1)^2 - 2(1) +1 \geq 0 \]
\[ 1 - 2 + 1 \geq 0\]
\[0 \geq 0 \]
Which is true. Thus, because we know that the left side of the inequality is greater than or equal to $0$ at its minumum, then we can infer that the left side of the inequality is greater than or equal to $0$ at every value of $x$ and conclude that the inequality is true. 

\medskip

Now move on to the statement $\frac{a+b}{2} \geq \sqrt{ab}$ for all $a, b > 0$. Square both sides of the inequality: 
\[\frac{(a+b)^2}{4} \geq ab \]



\item
Prove that $\sqrt{10}$ is irrational and deduce that $\sqrt{2} + \sqrt{5}$ is also irrational.

\item
Prove, by induction or otherwise, that $\sum_{k=1}^n k^3 = \left(\sum_{k=1}^n k\right)^2$.

\item
Each card in a pack has a number on one side and a letter on the other. Four cards are
placed on the table:
\[ \boxed{2} \; \boxed{3} \; \boxed{A} \; \boxed{B}\]
You are permitted to turn just two cards in order to test the following hypothesis: a card that
has an even number on one side has a vowel on the other. Which two cards should you turn and why? 

\item
Which of the following relations on $\mathbb{N}$ are reflexive, which are symmetric and which are transitive:
\begin{itemize}
\item[(i)] $a \mid b$ (`$a$ divides $b$'),
\item[(ii)] $a \nmid b$ (`$a$ does not divide $b$'),
\item[(iii)] $a$ and $b$ have the same remainder after division by $2025$,
\item[(iv)] gcd$(a, b) > 2025$?
\end{itemize}

\end{enumerate}


\end{document}