\documentclass[11pt]{article}

\usepackage{amsfonts,amsmath,amssymb,latexsym,color,epsfig}
\setlength{\textheight}{22.5cm} \setlength{\textwidth}{6.7in}
\setlength{\topmargin}{0pt} \setlength{\evensidemargin}{1pt}
\setlength{\oddsidemargin}{1pt} \setlength{\headsep}{10pt}
\setlength{\parskip}{1mm} \setlength{\parindent}{0mm}
\renewcommand{\baselinestretch}{1.03}

\newtheorem{theorem}{Theorem}
\newtheorem{conjecture}{Conjecture}
\newenvironment{proof}[1]
      {\medskip\noindent{\bf Proof #1}\hspace{1mm}}
      {\hfill$\Box$\medskip}


\title{\vspace{-0.7cm}Ma 1A - Problem Set 1}
\date{}

\begin{document}

\maketitle

\begin{enumerate}

\item
Prove that $x + \frac 1x \geq 2$ for all $x > 0$ and use this result to show that $\frac{a+b}{2} \geq \sqrt{ab}$ for all $a, b > 0$.

\bigskip 
\textbf{Proof: } First show that $x + \frac 1x \geq 2$ for all $x > 0$. Start by multiplying both sides of the inequality by $x$, which is ok because we only care about $x\geq 0$ and don't have to worry about sign switching or multiplying by a negative number: 
\[x^2 + 1 \geq 2x\]
Subtract $2x$ from both sides: 
\[x^2 -2x + 1 \geq 0 \]
Now let's find the minimum value of the left side of the inequality. Take the derivative and set it equal to zero: 
\[\tfrac{d}{dx} (x^2-2x+1 )= 2x -2 = 0 \]
\[2x = 2 \]
\[x = 1 \]
Thus we know that the minimum value of $x^2 - 2x + 1$ occurs at $x = 1$. Plug in $x = 1$ to the inequality to find the minimum value: 
\[(1)^2 - 2(1) +1 \geq 0 \]
\[ 1 - 2 + 1 \geq 0\]
\[0 \geq 0 \]
Which is true. Thus, because we know that the left side of the inequality is greater than or equal to $0$ at its minumum, then we can infer that the left side of the inequality is greater than or equal to $0$ at every value of $x$ and conclude that the inequality is true. 

\medskip

Now move on to the statement $\frac{a+b}{2} \geq \sqrt{ab}$ for all $a, b > 0$. Square both sides of the inequality: 
\[\frac{(a+b)^2}{4} \geq ab \]
Expand the top of the left side of the inequality: 
\[\frac{a^2 + 2ab + b^2}{4} \geq ab \]
Multiply both sides by 4: 
\[a^2 + 2ab + b^2 \geq 4ab\]
Subtract $2ab$ from both sides:
\[ a^2 + b^2 \geq 2ab\]
Divide both sides by $2ab$ ($a$ and $b$ are given to be always positive so no inequality sign switching takes place):
\[\frac{a^2 + b^2}{ab} \geq 2 \]
Split the fraction apart and simplify:
\[\frac{a^2}{ab} + \frac{b^2}{ab} \geq 2\]
\[\frac{a}{b}+ \frac{b}{a} \geq 2 \]
Plug in $x = \frac{a}{b}$:
\[x + \frac{1}{x} \geq 2 \] 
Which we know is true from the first part of this question. Thus, the statement $\frac{a+b}{2} \geq \sqrt{ab}$ for all $a, b > 0$ is true. 

\bigskip
\rightline{$\Box$}

\item
Prove that $\sqrt{10}$ is irrational and deduce that $\sqrt{2} + \sqrt{5}$ is also irrational.

\bigskip 
\textbf{Proof: } First prove that $\sqrt{10}$ is irrational. This will be a proof by contradiction. Assume to the contrary that $\sqrt{10}$ is rational and can be represented by an irreducible $\frac{p}{q}$ for $p,q \in \mathbb{Z}$, where $p$ and $q$ share no common factors. Then, 
\[\sqrt{10} = \frac{p}{q} \]
\[10 = \frac{p^2}{q^2}\]
\[p^2 = 10q^2 \]
 Note that if $p^2$ is a multiple of $10$, then $p$ is a multiple of $10$. Because $p$ is a multiple of 10, rewrite it as $10p_1, p_1 \in \mathbb{Z}$: 
\[100p_1 ^2 = 10 q^2\]
\[10p_1^2 = q^2   \]
Note that if $q^2$ is a multiple of $10$, then $q$ is also a multiple of $10$. Thus, both $p$ and $q$ share the common factor of $10$.
However, this is a contradiction, because we initially assumed that $p$ and $q$ shared no common factor. Thus, $\sqrt{10}$ cannot be rational and must instead be irrational. 

Now deduce that $\sqrt{2} + \sqrt{5}$ is also irrational. Assume to the contrary that $\sqrt{2} + \sqrt{5}$ is rational. Then, 
\[(\sqrt{2} + \sqrt{5})^2  = 2 + 2(\sqrt{2})(\sqrt{5}) + 5 = \]
\[ = 7 + 2\sqrt{10}\]
Let this supposedly rational number be represented by $q \in \mathbb{Q}$:
\[q = 7 + 2\sqrt{10}\]
Rearrange:
\[\frac{q-7}{2} = \sqrt{10}\]
However, $\frac{q-7}{2}$ should be rational because every number in it is rational, which creates a contradiction because we just showed that $\sqrt{10}$ is irrational. Thus, $\sqrt{2} + \sqrt{5}$ is also irrational.

\bigskip
\rightline{$\Box$}
\item
Prove, by induction or otherwise, that $\sum_{k=1}^n k^3 = \left(\sum_{k=1}^n k\right)^2$.

\bigskip\textbf{Proof: }

First examine the base case $n = 1$. Then, 
\[ \sum_{k=1}^1 k^3 = \left(\sum_{k=1}^1 k\right)^2 \]
\[ 1^3 = 1^2\]
\[ 1= 1\]

Now assume that $\sum_{k=1}^n k^3 = \left(\sum_{k=1}^n k\right)^2$ and show that the $n+1$ case holds true: 
\[ \sum_{k=1}^{n+1} k^3 = \left(\sum_{k=1}^{n+1} k\right)^2 \]
Seperate the last term of each series:
\[ \sum_{k=1}^{n} k^3 + (n+1)^3 = \left(\sum_{k=1}^n k + (n+1)\right)^2\]
Apply the inductive hypothesis: 
\[ \left(\sum_{k=1}^n k\right)^2 + (n+1)^3 = \left(\sum_{k=1}^n k + (n+1)\right)^2\]
Do binomial expansion on the right side:
\[ \left(\sum_{k=1}^n k\right)^2 + (n+1)^3 = \left( \sum_{k=1}^n k\right)^2 + 2\left( \sum_{k=1}^{n} k\right)(n+1) + (n+1)^2\]
Subtract $\left(\sum_{k=1}^n k\right)^2$ from both sides:
\[ (n+1)^3 = 2\left( \sum_{k=1}^{n} k\right)(n+1) + (n+1)^2\]
Follow in the footsteps of 10 year old Gauss and recognize that $\sum_{k=1}^{n} k = \frac{n(n+1)}{2}$:
\[ (n+1)^3 = 2(\frac{n(n+1)}{2})(n+1) + (n+1)^2\]
Now for some algebra:
\[ n^3 + 3n^2 +3n + 1 = n(n+1)(n+1)+ n^2 + 2n + 1 \]
\[ n^3 + 3n^2 +3n + 1 = (n^2+n)(n+1) + n^2 + 2n + 1\]
\[ n^3 + 3n^2 +3n + 1 = n^3 + n^2 + n^2 + n + n^2 + 2n + 1\]
\[ n^3 + 3n^2 +3n + 1 = n^3 + 3n^2 + 3n + 1\] 
And therefore the inductive case is proven. Because the base case and the inductive case are proven, the statement $\sum_{k=1}^n k^3 = \left(\sum_{k=1}^n k\right)^2$ is proven for all $n \geq 1$.

\rightline{$\Box$}
\item
Each card in a pack has a number on one side and a letter on the other. Four cards are
placed on the table:
\[ \boxed{2} \; \boxed{3} \; \boxed{A} \; \boxed{B}\]
You are permitted to turn just two cards in order to test the following hypothesis: a card that
has an even number on one side has a vowel on the other. Which two cards should you turn and why? 

\bigskip\textbf{Answer: } You only care about the statement that if a card has an even number on one side, then it has a vowel on the other. You don't care about what the card with 3 on one side has, for instance, because it doesn't have an even number on one side. Even if it ended up having a vowel on the other side, it would not invalidate our claim because it is not an ``iff" claim. Same goes for the card with the $B$ on the back, because we know it does not have a vowel on one side. There are only two cards that the hypothesis concerns: the card that has the even number $2$ on one side and the card that has the vowel $A$ on one side. We can use our two card turns to check these two cards and confirm or reject our hypothesis. 

\item
Which of the following relations on $\mathbb{N}$ are reflexive, which are symmetric and which are transitive:
\begin{itemize}
\item[(i)] $a \mid b$ (`$a$ divides $b$'),
\item[(ii)] $a \nmid b$ (`$a$ does not divide $b$'),
\item[(iii)] $a$ and $b$ have the same remainder after division by $2025$,
\item[(iv)] gcd$(a, b) > 2025$?
\end{itemize}

\end{enumerate}


\end{document}