\documentclass[11pt]{article}

\usepackage{amsfonts,amsmath,amssymb,latexsym,color,epsfig}
\setlength{\textheight}{22.5cm} \setlength{\textwidth}{6.7in}
\setlength{\topmargin}{0pt} \setlength{\evensidemargin}{1pt}
\setlength{\oddsidemargin}{1pt} \setlength{\headsep}{10pt}
\setlength{\parskip}{1mm} \setlength{\parindent}{0mm}
\renewcommand{\baselinestretch}{1.03}

\newtheorem{theorem}{Theorem}
\newtheorem{conjecture}{Conjecture}
\newenvironment{proof}[1]
      {\medskip\noindent{\bf Proof #1}\hspace{1mm}}
      {\hfill$\Box$\medskip}


\title{\vspace{-0.7cm}Ma 1A - Problem Set 2}
\date{}

\begin{document}

\maketitle

\begin{enumerate}

\item
For each of the following sets, decide whether they have each of a supremum, an infimum, a maximum or a minimum:
\begin{itemize}
\item[(a)] $\mathbb{Q} \cap [1, \sqrt{2}]$;

\bigskip
This is a finite set, so it will have a supremum and infimum. It also has a clear minimum at $1$. However, there is no maximum, because it is always possible to find a larger rational number $\frac{p}{q}$ that is smaller than $\sqrt{2}$. 

\item[(b)] $\{(-1)^n + 1/n : n \in \mathbb{N}\}$;

\bigskip
This set has no minimum, and a maximum at $1.5$ when $n=2$. This is still a finite set, so it will have a supremum and infimum. 

\item[(c)] $\cup_{n=1}^\infty [\frac{1}{2n}, \frac{1}{2n-1}]$.
This set will have a maximum at $n = 1$, but has no minimum. The set itself converges to $0$ and will not be infinite, so it has a supremum and infimum. 
\end{itemize}

\item
Suppose that $S$ and $T$ are nonempty subsets of $\mathbb{R}$ which are bounded above. Prove that $S \cup T$ is bounded above and $\sup (S\cup T) = \max(\sup S, \, \sup T)$.

\bigskip
\textbf{Proof: } First prove that $S \cup T$ is bounded above. If $S$ and $T$ are both bounded above, then suprema $sup(S)$ and $sup(T)$ both exist. Because they each exist for $S$ and $T$ respectively, both suprema exist for $S \cup T$. Call the max(sup($S$), sup($T$)) $B$. 

\medskip
Now consider a number $x \in S\cup T$. If $x \in S$, then $x \leq $ sup($S$) $\leq B$. If $x \in T$, then $x \leq $ sup($T$) $\leq B$. In any case, we see that $x \leq B$, so $B$ is an upper bound of $S\cup T$. 

\medskip Now prove that $\sup (S\cup T) = \max(\sup S, \, \sup T)$. Let $ C $ be an upper bound for $S\cup T$. This means that $C$ is also an upper bound for $S$ and $T$ individually. Then, by the definition of supremum, sup($S$) $\leq C$ and sup($T$) $\leq C$. Therefore, $B \leq U$, so $B$ is the least upper bound of $S\cup T$ and is the supremum of this set. 

\bigskip
\rightline{$\Box$}

\item
Prove that there exists a unique real number $a$ such that $a^3 = 2$. 

\bigskip
\textbf{Proof: } Let $S$ be the set $\{q \in \mathbb{R} : q^3 \leq 2\}$. We know that this set is not empty, because for instance $0$ and $1$ are clearly in the set. We also know the set has an upper bound, because $2^3 = 8 > 2$. 

\medskip
Now let's find the least upper bound $a = $ sup($S$). By the continuity axiom, $a$ exists. We will show that $a^3 = 2$ by showing that $a^3 < 2$ and $a^3>2$ create contradictions. 

First examine the case that $a^3 < 2$. Then, there exists an $\epsilon $ such that $ (a+ \epsilon)^3 < 2$, which creates a contradiction with the idea that $a$ is a supremum of $S$. 

Next examine the case that $a^3 > 2$. Then, there exists an $\epsilon $ such that $ (a- \epsilon)^3 < 2$, which creates a contradiction with the idea that $a$ is a supremum of $S$.  Thus, because $a^3$ cannot be greater or less than $2$, we know that $a^3 = 2$, so there exists a real number such that $a^3 = 2$. This number is also unique because $a^3$ increases monotonically. 

\rightline{$\Box$}


\item
Show that each of the following sequences $(a_n)_{n=1}^\infty$ converges to $0$:
\begin{itemize}
\item[(a)] $\frac{n+1}{n^2 + 1}$;

\medskip
Note that it is impossible for $\frac{n+1}{n^2 + 1}$ to be less than $0$, because no negative number can ever enter the fraction. Also note that it is monotonically decreasing because $n^2$ grows faster than $n$ and the sequence is proportional to $\frac{n}{n^2}$. Thus, the sequence is bounded and has a limit $A = lim_{n\rightarrow \infty}(\frac{n+1}{n^2 + 1})$.

Now let's use the squeeze theorem. We know that
\[\frac{n+1}{n^2+1} \leq \frac{n+1}{n^2} \]
because $n^2+1 \geq n^2$. Now do some algebra on the upper bound:
\[ \frac{n+1}{n^2}  = \frac{n}{n^2} + \frac{1}{n^2} =  \frac{1}{n} + \frac{1}{n^2} \leq \frac{1}{n}+\frac{1}{n} = \frac{2}{n}\]
We know that $\frac{2}{n} \rightarrow 0$ as $n \rightarrow \infty$, so by squeeze theorem $A = 0$.

\rightline{$\Box$}

\item[(b)] $2^{-n} \sin(n^3)$;

\medskip
We can again apply the squeeze theorem. We know that $-1 \leq sin(n^3) \leq 1$, and that $2^{-n} = \frac{1}{2^n} \leq \frac{1}{n}$. Thus, we can assert that 
\[-\frac{1}{n} \leq 2^{-n} \sin(n^3) \leq \frac{1}{n} \]
Because both $-\frac{1}{n}$ and $\frac{1}{n}$ converge to $0$, by squeeze theorem so does $2^{-n} \sin(n^3)$.

\rightline{$\Box$}

\item[(c)] $\sqrt{n} - \sqrt{n-1}$.

\medskip
We know that $\sqrt{n} - \sqrt{n-1}$ cannot be less than $0$ because $\sqrt{n-1} < \sqrt{n}$ for all $n\in\mathbb{N}$, and the square root function never outputs a negative number. 

Next, we can find some $n$ such that $|a_n - A| < \epsilon $ (we can disregard the absolute value because this expression is always positive):
\[a_n - 0 < \epsilon \Rightarrow \sqrt{n} - \sqrt{n-1} < \epsilon \]
Now multiply both sides by the conjugate (which is always positive):
\[n-(n-1) < \epsilon(\sqrt{n} + \sqrt{n-1} )\]
\[ 1 < \epsilon(\sqrt{n} + \sqrt{n-1} )\]
\[\frac{1}{\sqrt{n} + \sqrt{n-1} } < \epsilon\]
Take the reciprocal and switch the inequality sign:
\[ \sqrt{n} + \sqrt{n-1}  > \frac{1}{\epsilon}\]
Notice that $\sqrt{n} < \sqrt{n}+\sqrt{n+1}  $, so if we find an $n$ such that:
\[ \sqrt{n}  > \frac{1}{\epsilon}\]
then that $n$ also works out for 
\[ \sqrt{n} + \sqrt{n-1}  > \frac{1}{\epsilon}\] 
Thus, we know that 
\[ n = \frac{1}{\epsilon^2} \]
Satisfies $n$ such that $|a_n - A| < \epsilon $, where $A = 0$. 

\rightline{$\Box$}



\end{itemize}


\item
Prove or disprove that if $(a_n)_{n=1}^\infty$ is a sequence such that the subsequence $(a_{kn})_{n=1}^\infty$ converges for each $k = 2, 3, \dots$, then $(a_n)_{n=1}^\infty$ also converges.

Consider a sequence $(a_n)_{n=1}^\infty$, where for every prime $n$, $a_n = C$. For humor's sake, let the constant $C = 67$.  For every other $n$, $a_n = \frac{1}{n}$. Then, this function would be a counterexample to the statement in the question. This sequence will not converge because there are infinite primes, so it will be impossible to find an $N$ where all $n>N$ satisfies $|a_n - A| < \epsilon $ for any $\epsilon$. However, because $a_kn$ for $k \geq2$ will never include the prime-indexed numbers in the sequence, all $a_kn$ will converge like $\frac{1}{n}$.

\end{enumerate}


\end{document}