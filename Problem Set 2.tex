\documentclass[11pt]{article}

\usepackage{amsfonts,amsmath,amssymb,latexsym,color,epsfig}
\setlength{\textheight}{22.5cm} \setlength{\textwidth}{6.7in}
\setlength{\topmargin}{0pt} \setlength{\evensidemargin}{1pt}
\setlength{\oddsidemargin}{1pt} \setlength{\headsep}{10pt}
\setlength{\parskip}{1mm} \setlength{\parindent}{0mm}
\renewcommand{\baselinestretch}{1.03}

\newtheorem{theorem}{Theorem}
\newtheorem{conjecture}{Conjecture}
\newenvironment{proof}[1]
      {\medskip\noindent{\bf Proof #1}\hspace{1mm}}
      {\hfill$\Box$\medskip}


\title{\vspace{-0.7cm}Ma 1A - Problem Set 2}
\date{}

\begin{document}

\maketitle

\begin{enumerate}

\item
For each of the following sets, decide whether they have each of a supremum, an infimum, a maximum or a minimum:
\begin{itemize}
\item[(a)] $\mathbb{Q} \cap [1, \sqrt{2}]$;

\bigskip
This is a finite set, so it will have a supremum and infimum. It also has a clear minimum at $1$. However, there is no maximum, because it is always possible to find a larger rational number $\frac{p}{q}$ that is smaller than $\sqrt{2}$. 

\item[(b)] $\{(-1)^n + 1/n : n \in \mathbb{N}\}$;

\bigskip
This set has no minimum, and a maximum at $1.5$ when $n=2$. This is still a finite set, so it will have a supremum and infimum. 

\item[(c)] $\cup_{n=1}^\infty [\frac{1}{2n}, \frac{1}{2n-1}]$.
This set will have a maximum at $n = 1$, but has no minimum. The set itself converges to $0$ and will not be infinite, so it has a supremum and infimum. 
\end{itemize}

\item
Suppose that $S$ and $T$ are nonempty subsets of $\mathbb{R}$ which are bounded above. Prove that $S \cup T$ is bounded above and $\sup (S\cup T) = \max(\sup S, \, \sup T)$.



\item
Prove that there exists a unique real number $a$ such that $a^3 = 2$.

\item
Show that each of the following sequences $(a_n)_{n=1}^\infty$ converges to $0$:
\begin{itemize}
\item[(a)] $\frac{n+1}{n^2 + 1}$;
\item[(b)] $2^{-n} \sin(n^3)$;
\item[(c)] $\sqrt{n} - \sqrt{n-1}$.
\end{itemize}


\item
Prove or disprove that if $(a_n)_{n=1}^\infty$ is a sequence such that the subsequence $(a_{kn})_{n=1}^\infty$ converges for each $k = 2, 3, \dots$, then $(a_n)_{n=1}^\infty$ also converges.

\end{enumerate}


\end{document}